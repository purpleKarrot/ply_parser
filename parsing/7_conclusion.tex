\section{Conclusion}

Two parsing libraries for PLY files have been analyzed. Neither of them could
provide the flexibility, ease of use and safety to make it usable as a file
loader for a distributed rendering application.

The proposed library does not sacrifice type safety while being extremely
generic. Using it does not require the registration of any callback functions
and is thus much less complex  than using libply.

The implementation itself is extremely complex. While completely abstracted from
the programmer, the library needs to handle expressions for all possible
property types. This provides very high flexibility. The impact in performance
of this complexity still has to be investigated. It is also questionable whether
this high flexibility is really required.

As stated before, implementing a loader for a given PLY file is an easy task
compared to implementing a reusable parser library. A reusable parser library
provides worse performance compared to a specifically tailored parser and
negligible benefit: It still fails to load files that contain different content
than the client expects.

An alternative approach would be to write a program that creates a parser
library for a specific file by evaluating its header. This parser library
generator then could be included in the build process of a rendering
application. It would facilitate the development of PLY parsers, while still
providing optimal performance.

