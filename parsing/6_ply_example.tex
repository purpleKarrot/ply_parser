\section{PLY Parser}

The PLY file format was developed at Stanford University. The original goal of
the PLY format was to ``provide a format that is simple and easy to implement,
but that is general enough to be useful for a wide range of models''. The format
is actually general to such an extent, that it is not possible to make any
assertion about the content of a file until the file header has been parsed.
Consequently, it is impossible to support the full PLY format in a type-safe
manner.

A PLY-file constist


Writing a loader for a given PLY file is a relatively easy task. The types of
the content are known and the file header is used merely as a check that the
right file is opened. This loader can provide very high efficiency, but fails
loading any other file.

\begin{lstlisting}[float=*,frame=tb,caption=EBNF Grammar of the PLY file header]
ply      ::= "ply" EOL "format" format DOUBLE EOL element* "end_header" EOL
element  ::= "element" STRING INT EOL property*
property ::= "property" (list | scalar) STRING EOL
list     ::= "list" size scalar
format   ::= "ascii" | "binary_little_endian" | "binary_big_endian"
size     ::= "uint8" | "uint16"| "uint32" | "uint64"
scalar   ::= size | "int8" | "int16" | "int32" | "int64" | "float32" | "float64"
\end{lstlisting}

\begin{lstlisting}[float=*,frame=tb,caption=C++ Grammar of the PLY file header]
ply      %= "ply" > eol > "format" > format > double_ > eol > *element > "end_header" > eol;
element  %= "element" > *(char_ - int_) > int_ > eol > *property;
property %= "property" > (list | scalar) > *(char_ - eol) > eol;
list     %= "list" > size > scalar;
format   %= "ascii" | "binary_little_endian" | "binary_big_endian";
size     %= "uint8" | "uint16"| "uint32" | "uint64";
scalar   %= size | "int8" | "int16" | "int32" | "int64" | "float32" | "float64";
\end{lstlisting}

\subsection{Example use of the library}

Given you want to parse the vertex element of PLY files into a struct composed
of three Cartesian coordinates and three color components. To use this struct in
the proposed library, you need to adapt it as a Boost.Fusion sequence as shown
in Listing 2. Note that no change is required to the struct declaration itself.

\begin{lstlisting}[frame=tb,caption=Adapting a vertex struct]
struct vertex {
	float x, y, z;
	unsigned char red, green, blue;
};

BOOST_FUSION_ADAPT_STRUCT(vertex,
	(float, x)
	(float, y)
	(float, z)
	(unsigned char, red)
	(unsigned char, green)
	(unsigned char, blue)
)
\end{lstlisting}

The actual use of the library is shown in Listing 3. A \texttt{ply\_loader}
object is created and given the filename of the file to parse. The
\texttt{ply\_loader} can be configured by two \emph{policies}. Information about
policy base programming can be found in \cite{Alexandrescu:2001:MCD:377789}. The
\emph{error reporting policy} controls the detail level of errors to be reported
while the \emph{decompression policy} controls the stream decompression.

The two policies affect the iterator type that is used internally. The detailed
error reporting policy requires the current line of the file to be stored. When
this behavior is not desired, it can be turned off simply by using
\texttt{error\_simple} instead. The optional file decompression policy enables
bzip2, gzip and zlib decompression of streams. No decompression is performed
when this template parameter is omitted.

\begin{lstlisting}[float=*, frame=tb, caption=Example use of the library]
ply_loader<error_detailed, decompress_gzip> loader("bunny.ply.gz");
std::vector<vertex> vertices;
loader.parse_element(0, std::back_inserter(vertices));
\end{lstlisting}

The vertex definition of the Stanford Bunny's file header is shown in Listing
4. The properties do not match the declaration of the struct in Listing 2. The
library constructs a grammar at runtime that matches five consecutive floats
while the first three values are assigned to the coordinates, and the remaining
two are silently ignored. The colors will not be set. The colors can be set to a
default value, by defining a default constructor in the vertex struct of Listing
2.

\begin{lstlisting}[frame=tb,
  caption=Stanford Bunny's definition of the vertex element]
element vertex 35947
property float x
property float y
property float z
property float confidence
property float intensity
\end{lstlisting}

