\section{PLY Parser}

The PLY file format was developed at Stanford University. The original goal of
the PLY format was to ``provide a format that is simple and easy to implement,
but that is general enough to be useful for a wide range of models''. The format
is actually general to such an extent, that it is not possible to make any
assertion about the content of a file until the file header has been parsed.
Consequently, it is impossible to support the full PLY format in a type-safe
manner.

In applications where type safety is demanded, the to be parsed type is
precisely defined. This type information combined with the parsed header can be
used to build a parser to read the file body.

\begin{lstlisting}[float=*,label=plygrammar,frame=tb,caption=EBNF Grammar of the PLY file header]
ply      ::= "ply" EOL "format" format DOUBLE EOL element* "end_header" EOL
element  ::= "element" STRING INT EOL property*
property ::= "property" (list | scalar) STRING EOL
list     ::= "list" size scalar
format   ::= "ascii" | "binary_little_endian" | "binary_big_endian"
size     ::= "uint8" | "uint16"| "uint32" | "uint64"
scalar   ::= size | "int8" | "int16" | "int32" | "int64" | "float32" | "float64"
\end{lstlisting}

The content of the PLY file body is specified in the file header whose EBNF
grammar is shown in listing \ref{plygrammar}. It consists of multiple
\emph{element} definitions that consist of a name, a number specifying the size
and multiple \emph{property} definitions. Properties can be lists or single
types and provide a name too. The file body contains all the element blocks and
its properties one after another, each element block has the amont of elements
specified by the size. In case the file body is in ASCII format, each property
is terminated by a carriage return. Listing \ref{cube} shows a cube in PLY
format.

\begin{lstlisting}[label=cube,frame=tb,caption=PLY definition of a cube]
ply
format ascii 1.0
comment author: anonymous
comment object: another cube
element vertex 8
property float32 x
property float32 y
property float32 z
property red uint8
property green uint8
property blue uint8
element face 7
property list uint8 int32 vertex_index
element edge 5
property int32 vertex1
property int32 vertex2
property uint8 red
property uint8 green
property uint8 blue
end_header
0 0 0 255 0 0
0 0 1 255 0 0
0 1 1 255 0 0
0 1 0 255 0 0
1 0 0 0 0 255
1 0 1 0 0 255
1 1 1 0 0 255
1 1 0 0 0 255
3 0 1 2
3 0 2 3
4 7 6 5 4
4 0 4 5 1
4 1 5 6 2
4 2 6 7 3
4 3 7 4 0
0 1 255 255 255
1 2 255 255 255
2 3 255 255 255
3 0 255 255 255
2 0 0 0 0
\end{lstlisting}

\subsection{Reused and reusable parts}

The example implementation of the PLY reader contains parts that are not
specific to parsing PLY files, but can be reused to implement parsers that load
files with similar content but different grammar:

\begin{itemize}
  \item A parser class that wraps a grammar and a range of \emph{Forward
  Iterator}s. The class provides an invocation function that parses the
  grammatical expression once.
  \item A wrapper class around such a parser that models the \emph{Input
  Iterator} concept. The wrapped parser is invocated each time the iterator is
  incremented. It allows to copy parsed data to any output while parsing on the
  fly.
  \item A model of \emph{Output Iterator} that wraps another \emph{Output
  Iterator} and two indices defining the \emph{range of interest}. Outside the
  range of interest the wrapped iterator is not incremented and assignments are
  silently consumed. This class can be used to extract chunks from a stream.
\end{itemize}

A few other reusable components that have been used in the implementation
already are available in different libraries:

\begin{itemize}
  \item Streams that decompress data on the fly are provided by Boost.Iostreams.
  \item A class that wraps an \emph{Input Iterator} range and models the
  \emph{Forward Iterator} concept is provided by Boost.Spirit.
  \item An iterator class that stores information about the current position to
  provide helpful error messages when parsing fails is provided by Boost.Spirit.
\end{itemize}

\subsection{Implementation}

On the highest abstraction level, the PLY file reader is implemented as a
template class that is controlled by two policies: An error reporting policy and
a compression policy. Information about policy base programming can be found in
\cite{Alexandrescu:2001:MCD:377789}. The two policies affect the behaviour of
the class. The error reporting policy controls the detail of error messages.
When parsing fails, an exception will be thrown. When using
\texttt{error\_detailed}, the failure message will contain the line and column
numbers where the parsing error occured. When this behaviour is not desired,
i.e. when performance is critical or the file body is binary, it can be turned
off simply by using \texttt{error\_simple} instead. The optional file
decompression policy enables bzip2, gzip and zlib decompression of streams.
No decompression is performed when this template parameter is omitted.
The constructor of the class takes a filepath to the to PLY file to be parsed.

Internally, the two policies affect the iterator type that is used. The iterator
type is wrapped in several layers. On the lowest layer, the file specified by
the path is opened and a \texttt{istreambuf_iterator} is created. Depending on
the decompression policy, the stream buffer of the file is used directly, or a
\texttt{filtering_istreambuf} is initialized with a decompressor plus the file
and then used as a stream buffer.

Input stream buffer iterators model the \emph{Input Iterator} concept. Qi
however requires an \emph{Forward Iterator} for certain backtracking. For this
purpose, Qi provides the class \texttt{multi\_pass}, which wraps any input
iterator and exposes a forward iterator. This class is used to wrap the stream
buffer iterator.

The detailed error reporting policy requires the current line of the file to be
stored. To archive this, the forward iterator is wrapped by a
\texttt{position\_iterator}. This iterator class is also provided by Spirit.

%http://boost-spirit.com/home/articles/qi-example/tracking-the-input-position-while-parsing/

Once the iterators are all set up, the header of the PLY file is parsed. Parsing
the header is implemented with Qi. Note the similarity of the EBNF description
in listing \ref{plygrammar} and the C++ implementation in listing
\ref{plyspirit}.

\begin{lstlisting}[label=plyspirit, float=*,frame=tb,caption=C++ Grammar of the PLY file header]
ply      %= "ply" > eol > "format" > format > double_ > eol > *element > "end_header" > eol;
element  %= "element" > *(char_ - int_) > int_ > eol > *property;
property %= "property" > (list | scalar) > *(char_ - eol) > eol;
list     %= "list" > size > scalar;
format   %= "ascii" | "binary_little_endian" | "binary_big_endian";
size     %= "uint8" | "uint16"| "uint32" | "uint64";
scalar   %= size | "int8" | "int16" | "int32" | "int64" | "float32" | "float64";
\end{lstlisting}

The dynamic grammar construction happens when the user reads a block of
elements. The \texttt{parse\_element} member function takes an
\emph{Output Iterator} parameter. This gives a very high flexibility to the
application developer because it allows writing the result into container
stuctures such as \texttt{std::vector} or simple arrays, but also custom targets
such as console output. The library can deduce the type of the result from this
iterator and use it to generate a parser.

The PLY parser library now has the dynamic description of the element block
parsed from the header as well as the static result type deduced from the
iterator parameter. I now follows the parser generation described in section 5.
The resulting parser is wrapped into a class that models the
\emph{Input Iterator} concept. Parsing the whole block of elements then boils
down to a single call to \texttt{std::copy}.

In case parsing fails, an exception is thrown. When the detailed error reporting
policy is used, this exception is caught by the PLY parser, the failure message
is extended by position information and the exception is rethrown. This produces
error messages such as the one shown in listing \ref{error}.

\begin{lstlisting}[label=error,frame=tb,caption=detailed parsing error message]
Parsing error in file 'bunny.ply' line 27:
4 27 foo 13 42
     ^-- here
\end{lstlisting}


\subsection{Example use of the library}

Given you want to parse the vertex element of PLY files into a struct composed
of three Cartesian coordinates and three color components. To use this struct in
the proposed library, you need to adapt it as a Boost.Fusion sequence as shown
in Listing \ref{adapt}. Note that no change is required to the struct
declaration itself.

\begin{lstlisting}[label=adapt,frame=tb,caption=Adapting a vertex struct]
struct vertex
{
	float x;
	float y;
	float z;
	unsigned char red;
	unsinged char green;
	unsigned char blue;
};

BOOST_FUSION_ADAPT_STRUCT(vertex,
	(float, x)
	(float, y)
	(float, z)
	(unsigned char, red)
	(unsigned char, green)
	(unsigned char, blue)
)
\end{lstlisting}

The vertex definition of the Stanford Bunny's file header is shown in Listing
\ref{bunny}. The properties do not match the declaration of the struct in
Listing \ref{adapt}. The library constructs a grammar at runtime that matches
five consecutive floats while the first three values are assigned to the
coordinates, and the remaining two are silently ignored. The colors will not be
set. The colors can be set to a default value, by defining a default constructor
in the vertex struct of Listing \ref{adapt}.

\begin{lstlisting}[label=bunny, frame=tb,
  caption=Stanford Bunny's definition of the vertex element]
element vertex 35947
property float x
property float y
property float z
property float confidence
property float intensity
\end{lstlisting}

The actual use of the library is shown in Listing \ref{use}. The example uses
detailed error reporting and gzip decompression. Note that the filename
indicates that a gzip compressed file is given. The \texttt{parse\_element()}
function parses the block of elements with index 0 into a vector of the vertex
type from Listing \ref{adapt} using a back insert iterator. From this iterator,
the sequence type is deduced to generate an appropriate grammar. Passing
\texttt{vertex*} instead of a back insert iterator would work identically, since
\texttt{vertex*} also models the Output Iterator concept and the same type can
be deduced.

\begin{lstlisting}[label=use, frame=tb, caption=Example use of the library]
ply_loader<error_detailed, decompress_gzip>
	loader("bunny.ply.gz");
std::vector<vertex> vertices;
loader.parse_element(0,
	std::back_inserter(vertices));
\end{lstlisting}

