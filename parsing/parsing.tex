\documentclass[a4paper, parskip=half, twocolumn]{scrartcl}
%
\usepackage{geometry}
\geometry{a4paper,left=25mm,right=25mm, top=25mm, bottom=25mm} 
\setlength{\columnsep}{10mm}

%\usepackage{a4wide}
\usepackage[utf8]{inputenc}
\usepackage[T1]{fontenc}
\usepackage{times}

\usepackage[numbers]{natbib}
\usepackage{csquotes}

\usepackage{listings}
\lstset{basicstyle=\ttfamily\scriptsize\mdseries}

\usepackage{hyperref}
\hypersetup{colorlinks=true}
\urlstyle{same}

\title{Type-Safe Parsing with a Dynamic Grammar}

\author{Daniel Pfeifer,
\href{mailto:daniel@pfeifer-mail.de}{daniel@pfeifer-mail.de} \\
Department of Informatics (IFI), University of Zurich}

\date{\today}

\begin{document}

\maketitle

\begin{abstract}
In computer science, parsing is the process of analyzing a sequence of tokens
(words, bytes or even single bits) with respect to a formal grammar. The way
the analyzed data is processed is called semantics.

Parsers may be crafted by hand or generated by a tool that generates source code
from specification in a formal language. Both of these approaches require a
complete knowlegde of the data to be parsed.

In this paper I am going to present an approach of building a parser from a
dynamic grammar and static semantics. This allows to specify the data to be
parsed at runtime, but still provides static type safety. As an example
application, I wrote a parser for the PLY file format using the techniques
described in this paper.
\end{abstract}

\section{Introduction}

In the context of data storage and transmission, parsing plays a fundamental
role. Whenever a file is loaded from a disk or a network protocol is
interpreted, the file or stream content needs to be parsed and converted into a
data structure or object.

In simple cases, the file content is an exact representation of the data
structure. More general file formats and protocols provide a \emph{header}
information that describes the content of the \emph{body}.

A generic parser should be able to parse the body of a file or network packet,
with respect to the information from the respective header, thus generate a
parser from the formal grammar specified by the header. Because knowledge about
this grammar is deferred until runtime, let us call it a \emph{dynamic grammar}.

The application developer that uses a parser library specifies a class or data
structure that should be initialized by the parser. The parser has to initialize
this structure, no matter what the contents of the file really are. Because this
structure defines a type and C++ is a statically typed language, the semantics
of the parser is \emph{static}.

In this paper I present an approach of building a general parser from a dynamic
grammar and static semantics. As an example application, I demonstrate a parser
for the Polygon File Format (PLY) using the techniques described in this paper.


\section{Related Work}

\cite{Grune:1990:PTP:130365} defines the term \emph{parsing} as \enquote{the
process of structuring a linear representation in accordance with a given
grammar}. From a given grammar, there are generally three ways to generate a
parser:

The grammar can be passed to a team of programmers that craft the parser by
hand, potentially following a specification. It is advisable, though not
strictly required, to formally specify the grammar. This approach is error prone
and requires acceptance testing.

If the grammar is formally specified, the parser generation can be automated.
Tools like ANTLR or Lex/YACC generate source code from a grammatical
description \cite{Johnson75yacc:yet, Parr95antlr:a}. This approach is more
flexible and less prone to programming errors compared to a hand crafted
solution. If a modification of the parser's behavior is desired, all that needs
to be tweaked is the description of the grammar. The code itself is then
automatically regenerated from the specification.

Yet another approach is to embed the formal grammar directly into the source
code in a domain specific embedded language (e.g. Qi from the Boost.Spirit
library).

What all of these approaches have in common, is that they operate on the source
code level. It follows, that they all provide static type-safety, but also
static behavior! In case our knowledge about the grammar is deferred until
runtime, we cannot depend on generated source code, hand crafted or not.

Regular expression processors such as the ones found in Awk, Perl, Ruby, and Tcl
and described in \cite{Friedl:2006:MRE:1209014} are typical examples of parsers
with dynamic behavior. The processor is initialized at runtime with a string if
text that describes the grammar in regualar expression syntax. Instead of
generating source code, the regular expression processor tokenizes an input
string of text into substrings.

The limitation to strings of text in the input and output frustrates the use of
regular expressions when type-safety is desired.


\section{Generic Parsing}

Say we want to parse an ASCII file that in each line has three floating point
numbers specifying a position in cartesian coordinates followed by three
integers specifying a color in RGB-format. The data should be loaded into a
vertex type that we specify as:

\begin{lstlisting}[frame=tb,label=vertex,caption=Vertex Example]
struct vertex
{
  float x, y, z;
  uchar b, g, r;
};
\end{lstlisting}

Implementing a parser for this file format would be straight forward. But if we
now take a file that provides the position in double precision, or two
dimensional positions, or color with an alpha channel, or binary data, our
parser would crash or yield undefined behavior.

If we augment the data file with some header information that describes the
content of the file, there are multiple ways to handle this additional
information.

The header data could be used merely as a check that the file content exacly
matches the expected type. If the data does not match, an error is reported and
parsing the file is aborted. This is better than a crash, but under certain
conditions unneccessarily strict. It would be safe to just ignore data that is
present in the file stream but not requested by the client. Similarly, data that
is required by the client but not existent in the file could be initialized to a
default; data that is present in a different type, encoding or order than
expected could be converted or reordered. There is generally no need to not
support files or streams that do not exacly match.

Alternatively, a binary block of memory can be allocated depending on the
information contained in the file header. The parser then fills this block with
the parsed data. The client may access the data by adding an offset to the start
address and reinterpret the type with a casting operator. This approach relaxes
all type checking mechanisms of the compiler and puts the burdon of using the
types correctly on the application developer. Apart from providing no
type-safety at all, this approach also requires all parsed data to be bufferd.

In yet another approach the application developer \enquote{tells} the library
what data she wants and where it should be written to. The library parses the
file, converts the parsed data on the fly, and writes it into the area that the
client provides. In the C programming language, we would register the result
values as a combination of \texttt{void*} pointers and enumerated values
specifying the type. Again, there is no type safety and the burdon of using the
appropriate enums lies on the application developer.

In C++, we can waive the parameter for the type if we pass the result value by a
mutable reference. By the use of Template Metaprogramming techniques described
in \cite{Abrahams:2004:CTM:1044941} we can deduce both the address and the type
of of the value that should be initialized by the parser. This not only
disburdens the application developer, but also the library developer from
reinterpreting the type of the passed parameter; thus the library can be
implemented in a type safe manner.

And we can even go a step further: From a result type such as the
\texttt{vertex} structure from Listing \ref{vertex} we can deduce the amount,
the types and even the names of its members using pure template metaprogramming. 
This will allow us to implement a parsing library that provides an interface
similar to the one presented in Listing \ref{interface}. Contrary to the
example, a real library should generate the parser once and use it multiple
times.

\begin{lstlisting}[frame=tb,label=interface,caption=parser interface]
template<template Output>
void parse(const std::string& data,
  const input_t& input, Output& output)
{
  parser<Output> p(input);
  output = p.parse(data);
}
\end{lstlisting}

The callee on this function will pass three pieces of information through the
parameters. The data to be parsed is a string of text or more generally a range
of arbitrary data. The content of this data is described by the the
\texttt{input} parameter. This is usually a list of name-value pairs, where the
value enumerates the type of each element as shown in Listing \ref{element}.
Last but not least the \texttt{output} parameter specifies both the type and the
location of the result of the parsing procedure.

\begin{lstlisting}[frame=tb,label=element,caption=element type]
struct element
{
  enum { efloat, edouble, eint, ... } type; 
  std::string name;
};
typedef std::vector<element> input\_t;
\end{lstlisting}

The parser will be generated from the input list and the output type
defininition only. It will be used once or multiple times to parse the data
stream. Listing \ref{algorithm} shows the algorithm that is used to construct
the parser as a mixture of pseudo code and Backus-Naur Form (BNF).

First of all, the grammar is initialized with a parser that matches a zero
length string (epsilon). While iterating over all input elements, rules are
initialized and appended to the grammar. Optionally, the grammar may be
finalized in the end by appending a rule that matches delimiter symbols like
line endings.

\begin{lstlisting}[mathescape, frame=tb, label=algorithm,
                   caption=grammar construction algorithm]
GRAMMAR ::= $\epsilon$
foreach element i in input
  RULE ::= omit_parser(i.type)
  foreach o in Output
    if i.name equals o.name
      RULE ::= assign_parser(i.type, o.value)
    endif
  endforeach
  GRAMMAR ::= GRAMMAR' RULE
endforeach
GRAMMAR ::= GRAMMAR' EOL
\end{lstlisting}

To initialize a rule, we iterate over the elements of the output, checking for
each element whether the name equals the name that is expected in the input. If
the names match, the rule is initialized with a parser that matches a string
specified by the current input type and returns a value of the type specified by
the current output element into the right position. It cannot be guaranteed that
such an element exists. Therefore, all rules have to be initialized before with
parsers that match a string specified by the current input but do not return
anything, ignoring the result.

Implementation-wise, looping over the input sequence is a conventional
\texttt{for}-construct. The inner loop however needs to be unrolled at compile
time, using template metaprogramming, since it iterates over a sequence of types
rather than values.


\section{Generic Programming}

Generic programming is about writing code that is independent of the types of
objects being manipulated. The \texttt{memcpy()} function of the C standard
library is generalized at the price of type safety by the use of \texttt{void*}.
In C++, class and function templates make the generalization possible without
sacrificing type safety
\cite{Alexandrescu:2001:MCD:377789,Meyers:2005:ECS:1051335}.

The set of requirements consisting of valid expressions, associated types,
invariants and complexity guarantees is summarized as a \emph{concept}. A type
that satisfies the requirements is said to \emph{model the concept}. A concept
can extend the requirements of another concept, which is called
\emph{refinement} \cite{gregor08:devx_concepts}. The concepts used in the C++
Standard Library are documented at the SGI STL
site\footnote{\url{http://www.sgi.com/tech/stl/table_of_contents.html}}.


\section{Implementation}

Implementing the parser is realized using Qi from the Boost.Spirit library. One
improtant feature of Qi is that it allows to reassign rules as shown in listing
\ref{reassign}. In the second line we take a rule, copy it\footnote{otherwise
this operation generates a recursive parser}, append another terminal and assign
the resulting parser to the rule, overiding its previous value. 

\begin{lstlisting}[frame=tb,label=reassign,caption=reassigning rules]
RULE ::= TEMINAL1
RULE ::= RULE' TERMINAL2
\end{lstlisting}

Further used libraries are Boost.Fusion and Boost.MPL for template
metaprogramming. Also, some functional programming with Boost.Phoenix is
utilized.

As was described before, the parser will be generated from two components. The
first component is the type which the parser will return. We call this type
\texttt{Output}, starting with a capital letter to denote it is a generic type.
\texttt{Output} is required to model the \texttt{Sequence} Concept defined by
the Boost.Fusion library to allow us to inspect its elements with template
metaprogramming.

The second component is a dynamically sized list of \texttt{struct element}
shown in listing \ref{element}. This list is usually filled by parsing the
header information. We call this list \texttt{input\_t}, in all lower case, to
denote that its type is specificied.

For the parser class, we use \texttt{Output} as a template argument and pass
\texttt{input\_t} to the constructor. Internally, we create a rule for each
element of \texttt{input\_t}. Each rule will parse a type depending on the
\texttt{type} enumerator and it will have a semantic action depending on
\texttt{Output}. These rules will be chained by appending them to the grammar
as shown in listing \ref{reassign}.

In Qi, each nonterminal such as a rule has a signature, the same way as a
function has a signature. The signature specifies the semantic action of the
rule. Assigning a parser to a rule requires that the signature matches. 
Invoking a nonterminal yields in another nonterminal that inherits all passed
parameters as attributes. Given a nonterminal with a signature
\texttt{float(int)} invoked with an \texttt{int} argument yields a nonterminal
with a signature \texttt{float()}.

The parser construction follows the algorithm shown in listing \ref{algorithm}.
However, since rules must hold terminals and subrules by reference, it is not
possible to return subrules from a factory function, which would be the
preferred method to generate objects depending on an enumerated value. Further,
semantic actions require the types of the rules to be specified at compile time.

It follows that the parser needs to hold all rules that might require a semantic
action by value. This is exactly one rule per element in \texttt{input\_t} plus
one for each element in \texttt{Output} whose name matches an element in
\texttt{input\_t}. Wether the names match is unknown at compile time, so we need
to provide rules for all elements of \texttt{Output}, accepting the fact that
they might not be used.

The amount of rules depending on \texttt{input\_t} is dynamic, so we generate an
array of rules at runtime. Each of these rules may or may not initialize any
element of \texttt{Output}. We therefore use \texttt{void(Output\&)} as a
signature for all of these rules.

To instanciate the rules for the elements of \texttt{Output}, we transform
the \texttt{Output} sequence into a rule sequence that contains a rule with a
signature of \texttt{void(T\&)} for each type \texttt{T} in \texttt{Output}.

%It cannot be guaranteed that such an element exists. Therefore,
%all subrules have to be initialized before with parsers that match a string
%specified by the current input but do not return anything. From that follows
%that the subrule needs to be able to store a parser that optionally initializes
%a value, hence the \texttt{void(T\&)} signature: It would be impossible to
%dynamically change the signature form \texttt{void()} to \texttt{T()} when the
%names match.

Looking at the parser generation of listing \ref{algorithm} in more detail, we
iterate in a dynamic loop over the elements of the \texttt{input\_t} list. Each
element has a corresponding nonterminal called \emph{input rule} with a
signature of \texttt{void(Output\&)}. This rule is initialized with a parser
that parses but discards a value corresponding to the type enumeration of the
current element.

In a nested, static loop we iterate over the \texttt{Output} sequence. Each
element \texttt{T} of the \texttt{Output} sequence has a corresponding
nonterminal called \emph{output rule} with a signature of \texttt{void(T\&)}.
If and only if the name of the current element of the \texttt{Output} sequence
equals the name of the current element of the \texttt{input\_t} list, the
\emph{output rule} is initialized with a parser that parses a value
corresponding to the type enumeration of the current element of the
\texttt{input\_t} list and stores it in the local attribute. This attibute is
inherited by invoking the \emph{output rule} with a lazy element accessor of
\texttt{Output}, which yields a nonterminal with a signature
\texttt{void(Output\&)}. This nonterminal is assigned to the \emph{input rule},
overiding its previous value.

To append the \emph{input rule} to the grammar, it is invoked with the
\texttt{Output} value that will hold the result of the parser which yields a
nonterminal with a signature of \texttt{void()}. Concatenating two parsers with
a signature of \texttt{void()} yields a nonterminal with the same signature,
which makes it possible to assign the resulting nonterminal to the grammar.

Initializing the input and output rules with a parser is done via a function
similar to a factory. The rule that will hold the parser is passed as a mutable
reference instead of being initialized by the return value to avoid slicing.

%The grammar generation can be completely abstracted from the application
%developer. The information about where the result should be written to, gives
%the parser library the complete information about the type of the expected
%result. There is no need to require the application developer to register
%the structure and its elements directly to the parser library.

%Listing \ref{interface} shows an example of how a parser function could be
%implemented that parses a string into a user defined type \texttt{Output}. The
%content of the string is described by an input sequence of element descriptions.
%The grammar generation is hidden as an implementation detail that the
%application developer does not need to care about.


\section{PLY Parser}

The PLY file format was developed at Stanford University. The original goal of
the PLY format was to ``provide a format that is simple and easy to implement,
but that is general enough to be useful for a wide range of models''. The format
is actually general to such an extent, that it is not possible to make any
assertion about the content of a file until the file header has been parsed.
Consequently, it is impossible to support the full PLY format in a type-safe
manner.

A PLY-file constist


Writing a loader for a given PLY file is a relatively easy task. The types of
the content are known and the file header is used merely as a check that the
right file is opened. This loader can provide very high efficiency, but fails
loading any other file.

\begin{lstlisting}[float=*,frame=tb,caption=EBNF Grammar of the PLY file header]
ply      ::= "ply" EOL "format" format DOUBLE EOL element* "end_header" EOL
element  ::= "element" STRING INT EOL property*
property ::= "property" (list | scalar) STRING EOL
list     ::= "list" size scalar
format   ::= "ascii" | "binary_little_endian" | "binary_big_endian"
size     ::= "uint8" | "uint16"| "uint32" | "uint64"
scalar   ::= size | "int8" | "int16" | "int32" | "int64" | "float32" | "float64"
\end{lstlisting}

\begin{lstlisting}[float=*,frame=tb,caption=C++ Grammar of the PLY file header]
ply      %= "ply" > eol > "format" > format > double_ > eol > *element > "end_header" > eol;
element  %= "element" > *(char_ - int_) > int_ > eol > *property;
property %= "property" > (list | scalar) > *(char_ - eol) > eol;
list     %= "list" > size > scalar;
format   %= "ascii" | "binary_little_endian" | "binary_big_endian";
size     %= "uint8" | "uint16"| "uint32" | "uint64";
scalar   %= size | "int8" | "int16" | "int32" | "int64" | "float32" | "float64";
\end{lstlisting}

\subsection{Example use of the library}

Given you want to parse the vertex element of PLY files into a struct composed
of three Cartesian coordinates and three color components. To use this struct in
the proposed library, you need to adapt it as a Boost.Fusion sequence as shown
in Listing 2. Note that no change is required to the struct declaration itself.

\begin{lstlisting}[frame=tb,caption=Adapting a vertex struct]
struct vertex {
	float x, y, z;
	unsigned char red, green, blue;
};

BOOST_FUSION_ADAPT_STRUCT(vertex,
	(float, x)
	(float, y)
	(float, z)
	(unsigned char, red)
	(unsigned char, green)
	(unsigned char, blue)
)
\end{lstlisting}

The actual use of the library is shown in Listing 3. A \texttt{ply\_loader}
object is created and given the filename of the file to parse. The
\texttt{ply\_loader} can be configured by two \emph{policies}. Information about
policy base programming can be found in \cite{Alexandrescu:2001:MCD:377789}. The
\emph{error reporting policy} controls the detail level of errors to be reported
while the \emph{decompression policy} controls the stream decompression.

The two policies affect the iterator type that is used internally. The detailed
error reporting policy requires the current line of the file to be stored. When
this behavior is not desired, it can be turned off simply by using
\texttt{error\_simple} instead. The optional file decompression policy enables
bzip2, gzip and zlib decompression of streams. No decompression is performed
when this template parameter is omitted.

\begin{lstlisting}[float=*, frame=tb, caption=Example use of the library]
ply_loader<error_detailed, decompress_gzip> loader("bunny.ply.gz");
std::vector<vertex> vertices;
loader.parse_element(0, std::back_inserter(vertices));
\end{lstlisting}

The vertex definition of the Stanford Bunny's file header is shown in Listing
4. The properties do not match the declaration of the struct in Listing 2. The
library constructs a grammar at runtime that matches five consecutive floats
while the first three values are assigned to the coordinates, and the remaining
two are silently ignored. The colors will not be set. The colors can be set to a
default value, by defining a default constructor in the vertex struct of Listing
2.

\begin{lstlisting}[frame=tb,
  caption=Stanford Bunny's definition of the vertex element]
element vertex 35947
property float x
property float y
property float z
property float confidence
property float intensity
\end{lstlisting}


\section{Conclusion}

Two parsing libraries for PLY files have been analyzed. Neither of them could
provide the flexibility, ease of use and safety to make it usable as a file
loader for a distributed rendering application.

The proposed library does not sacrifice type safety while being extremely
generic. Using it does not require the registration of any callback functions
and is thus much less complex  than using libply.

The implementation itself is extremely complex. While completely abstracted from
the programmer, the library needs to handle expressions for all possible
property types. This provides very high flexibility. The impact in performance
of this complexity still has to be investigated. It is also questionable whether
this high flexibility is really required.

As stated before, implementing a loader for a given PLY file is an easy task
compared to implementing a reusable parser library. A reusable parser library
provides worse performance compared to a specifically tailored parser and
negligible benefit: It still fails to load files that contain different content
than the client expects.

An alternative approach would be to write a program that creates a parser
library for a specific file by evaluating its header. This parser library
generator then could be included in the build process of a rendering
application. It would facilitate the development of PLY parsers, while still
providing optimal performance.



\section{Approach}
\subsection{Motivating Example}
An example data structure is
given in Listing \ref{vertex}. This data structure directly affects the
semantics of the parser.

\begin{lstlisting}[frame=tb,label=vertex,caption=Vertex Example]
struct vertex
{
  float x, y, z;
  uchar b, g, r;
};
\end{lstlisting}

We have a file that contains a list of vertices. Each vertex consists of a
position and a color. The position is given as two double precision floating
point numbers and the color is in ARGB format, where each color channel is an
unsigned char. The list of vertices should be parsed, and written into an array
of the vertex struct introduced in Listing \ref{vertex}.

The parser should be able to match these seven numbers in the format that is
specified in the header, convert the first two numbers to float, assign the
values to the corresponding elements of the struct (taking reordering into
account), set the element z to a default and discard the alpha channel of the
color.

\subsection{Motivating Example}



\subsection{Motivating Example}





\section{Custom Approach}

The two reviewed libraries do not provide the safety and ease of use that are
desired for a parser to be used in a parallel rendering application. I propose
another approach, that is completely typesafe and requires only a minimum of
setup. It is still flexible enough to read from any kind on input like
compressed files or network streams.

My library uses the Qi parser library which allows to write grammars and format
descriptions using a format similar to Extended Backus Naur Form (EBNF) directly
in C++. The grammar to parse the PLY file header is summarized in Listing 1.


Building a parser from this grammar is straight forward with both Lex/YACC and
Qi. The grammar to parse the file body cannot be known at compile time and has
to be constructed after the header has been parsed. In this context Qi's
expression templates win over Lex/YACC.

Using my library, the user has to register neither types nor callback functions.
The only requirement is that the PLY elements are assigned to an object whose
type models the \emph{Sequence} concept of the Boost.Fusion library.

Any custom struct can be adapted by the macro
\texttt{BOOST\_FUSION\_ADAPT\_STRUCT} to fulfill this requirement without
modifying the struct itself . All further required information, even the
property names, are deduced by applying template metaprogramming techniques
described in \cite{Abrahams:2004:CTM:1044941} and the documentation of
Fusion\footnote{\url{http://www.boost.org/doc/libs/release/libs/fusion/}}.

\subsection{Generic Programming Techniques}

The function \texttt{std::copy()} copies a range of data of a generic type,
provided that the input parameters model \emph{Input Iterator} and the output
parameter models \emph{Output Iterator}. To extract a chunk, one can either
\begin{itemize}
  \item increment the input parameter to the desired begin of the chunk and then
  copy all data before the desired end or
  \item provide a custom output iterator type that skips data before the chunk.
\end{itemize}

Loading different chunks successively is a multipass algorithm. Multipass
algorithms require the iterators to model \emph{Forward Iterator} which is a
refinement of \emph{Input Iterator}.

Iterators of the C++ Standard Library that read from an input stream model the
\emph{Input Iterator} concept. The class \texttt{boost::spirit::multi\_pass} of
the Boost.Spirit library wraps any input iterator and models \emph{Forward
Iterator} by the use of reference counted buffers.

This class should be used with care. It is intended to support backtracking in
parsing grammars that contain alternatives. Improper use of this class can lead
to buffering the complete stream in the worst case.

\subsection{Reused and reusable parts}

The implementation of the library contains parts that are not specific to
parsing PLY files, but can be reused to implement parsers that load files with
similar content but different grammar:

\begin{itemize}
  \item A parser class that wraps a grammar and a range of \emph{Forward
  Iterator}s. The class provides an invocation function that parses the
  grammatical expression once.
  \item A wrapper class around such a parser that models the \emph{Input
  Iterator} concept. The wrapped parser is invocated each time the iterator is
  incremented. It allows to copy parsed data to any output while parsing on the
  fly.
  \item A model of \emph{Output Iterator} that wraps another \emph{Output
  Iterator} and two indices defining the \emph{range of interest}. Outside the
  range of interest the wrapped iterator is not incremented and assignments are
  silently consumed. This class can be used to extract chunks from a stream.
\end{itemize}

A few other reusable components that have been used in the implementation
already are available in different libraries:

\begin{itemize}
  \item Streams that decompress data on the fly are provided by Boost.Iostreams.
  \item A class that wraps an \emph{Input Iterator} range and models the
  \emph{Forward Iterator} concept is provided by Boost.Spirit.
  \item An iterator class that stores information about the current position to
  provide helpful error messages when parsing fails is provided by Boost.Spirit.
\end{itemize}

\bibliographystyle{plain}
\bibliography{bibliography}

\end{document}
