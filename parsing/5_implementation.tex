\section{Implementation}

Implementing the parser is realized using Qi from the Boost.Spirit library. One
improtant feature of Qi is that it allows to reassign rules as shown in listing
\ref{reassign}. In the second line we take a rule, copy it\footnote{otherwise
this operation generates a recursive parser}, append another terminal and assign
the resulting parser to the rule, overiding its previous value. 

\begin{lstlisting}[frame=tb,label=reassign,caption=reassigning rules]
RULE ::= TEMINAL1
RULE ::= RULE' TERMINAL2
\end{lstlisting}

Further used libraries are Boost.Fusion and Boost.MPL for template
metaprogramming. Also, some functional programming with Boost.Phoenix is
utilized.

As was described before, the parser will be generated from two components. The
first component is the type which the parser will return. We call this type
\texttt{Output}, starting with a capital letter to denote it is a generic type.
\texttt{Output} is required to model the \texttt{Sequence} Concept defined by
the Boost.Fusion library to allow us to inspect its elements with template
metaprogramming.

The second component is a dynamically sized list of \texttt{struct element}
shown in listing \ref{element}. This list is usually filled by parsing the
header information. We call this list \texttt{input\_t}, in all lower case, to
denote that its type is specificied.

For the parser class, we use \texttt{Output} as a template argument and pass
\texttt{input\_t} to the constructor. Internally, we create a rule for each
element of \texttt{input\_t}. Each rule will parse a type depending on the
\texttt{type} enumerator and it will have a semantic action depending on
\texttt{Output}. These rules will be chained by appending them to the grammar
as shown in listing \ref{reassign}.

In Qi, each nonterminal such as a rule has a signature, the same way as a
function has a signature. The signature specifies the semantic action of the
rule. Assigning a parser to a rule requires that the signature matches. 
Invoking a nonterminal yields in another nonterminal that inherits all passed
parameters as attributes. Given a nonterminal with a signature
\texttt{float(int)} invoked with an \texttt{int} argument yields a nonterminal
with a signature \texttt{float()}.

The parser construction follows the algorithm shown in listing \ref{algorithm}.
However, since rules must hold terminals and subrules by reference, it is not
possible to return subrules from a factory function, which would be the
preferred method to generate objects depending on an enumerated value. Further,
semantic actions require the types of the rules to be specified at compile time.

It follows that the parser needs to hold all rules that might require a semantic
action by value. This is exactly one rule per element in \texttt{input\_t} plus
one for each element in \texttt{Output} whose name matches an element in
\texttt{input\_t}. Wether the names match is unknown at compile time, so we need
to provide rules for all elements of \texttt{Output}, accepting the fact that
they might not be used.

The amount of rules depending on \texttt{input\_t} is dynamic, so we generate an
array of rules at runtime. Each of these rules may or may not initialize any
element of \texttt{Output}. We therefore use \texttt{void(Output\&)} as a
signature for all of these rules.

To instanciate the rules for the elements of \texttt{Output}, we transform
the \texttt{Output} sequence into a rule sequence that contains a rule with a
signature of \texttt{void(T\&)} for each type \texttt{T} in \texttt{Output}.

%It cannot be guaranteed that such an element exists. Therefore,
%all subrules have to be initialized before with parsers that match a string
%specified by the current input but do not return anything. From that follows
%that the subrule needs to be able to store a parser that optionally initializes
%a value, hence the \texttt{void(T\&)} signature: It would be impossible to
%dynamically change the signature form \texttt{void()} to \texttt{T()} when the
%names match.

Looking at the parser generation of listing \ref{algorithm} in more detail, we
iterate in a dynamic loop over the elements of the \texttt{input\_t} list. Each
element has a corresponding nonterminal called \emph{input rule} with a
signature of \texttt{void(Output\&)}. This rule is initialized with a parser
that parses but discards a value corresponding to the type enumeration of the
current element.

In a nested, static loop we iterate over the \texttt{Output} sequence. Each
element \texttt{T} of the \texttt{Output} sequence has a corresponding
nonterminal called \emph{output rule} with a signature of \texttt{void(T\&)}.
If and only if the name of the current element of the \texttt{Output} sequence
equals the name of the current element of the \texttt{input\_t} list, the
\emph{output rule} is initialized with a parser that parses a value
corresponding to the type enumeration of the current element of the
\texttt{input\_t} list and stores it in the local attribute. This attibute is
inherited by invoking the \emph{output rule} with a lazy element accessor of
\texttt{Output}, which yields a nonterminal with a signature
\texttt{void(Output\&)}. This nonterminal is assigned to the \emph{input rule},
overiding its previous value.

To append the \emph{input rule} to the grammar, it is invoked with the
\texttt{Output} value that will hold the result of the parser which yields a
nonterminal with a signature of \texttt{void()}. Concatenating two parsers with
a signature of \texttt{void()} yields a nonterminal with the same signature,
which makes it possible to assign the resulting nonterminal to the grammar.

Initializing the input and output rules with a parser is done via a function
similar to a factory. The rule that will hold the parser is passed as a mutable
reference instead of being initialized by the return value to avoid slicing.

%The grammar generation can be completely abstracted from the application
%developer. The information about where the result should be written to, gives
%the parser library the complete information about the type of the expected
%result. There is no need to require the application developer to register
%the structure and its elements directly to the parser library.

%Listing \ref{interface} shows an example of how a parser function could be
%implemented that parses a string into a user defined type \texttt{Output}. The
%content of the string is described by an input sequence of element descriptions.
%The grammar generation is hidden as an implementation detail that the
%application developer does not need to care about.

